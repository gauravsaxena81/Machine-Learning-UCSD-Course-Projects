\documentclass{acm_proc_article-sp}

\begin{document}

\title{Gender classification using Logistic Regression}

\numberofauthors{2} %  
\author{
% 1st. author
Suvir Jain,
% 2nd. author
Gaurav Saxena
}

\maketitle
\begin{abstract}
Gender classification is important in several applications like tailoring personalized experience of stores and gadgets, providing analytics data to companies about customers etc. We use 800 different features to train a Logisitc Regression model to achieve this task. We report the performance of Logistic Regression using Stochastic Gradient Descent and L-BFGS. Furthermore, to protect model from over-fitting, we use cross validation and $L_{2}$ Regularization. In our experiments with test data we found that our model was able to identify the correct labels 92\% of the time
\end{abstract}

\section{Introduction}

What is the problem?

Logistic Regression is a statistical classification model to determine the probabilities of the outcomes based on the occurrence of certain features. The process of deriving this model is called training the model. This process finds weights for the features such that their linear combination can be used to determine the 'most probable' outcome. However, linear models are unbounded and thus, they cannot be directly used to calculate the probabilities, which are bounded between 0 and 1. To achieve these bounds this process uses a squashing function which reduces the range of the linear function to be between 0 and 1.

To train this model a data set is used, which contains various observations of an experiment. Each observation of the experiment consists of the features and the outcome. For example, let us consider an observation of the the data set [1] for gender classification. Each observation of this data set consists of 800 features and their values for a person. It also contains a label to identify the person as male or female. This label acts as the outcome of the experiment.

Why is it interesting and important?
Why is it hard? (E.g., why do naive approaches fail?)

This data can be used to deduce the optimal values of the weights for the features by solving a system of non-linear equations in 800 variables. However, achieving this by mathematical methods may be intractable. Thus, numerical methods like Gradient Descent[2] are used to find optimal values. However, gradient descent for large data sets tend to be slow. This is because it has a time complexity of $O(n^{2}d)$[3] where n is the number of observations and d is the number of dimensions. Thus, we use other numerical methods like Stochastic Gradient Descent(SGD) and L-BFGS which are less accurate but are faster [4].

Although, numerical methods make hard mathematical problems tractable, their convergence is dependent on a number of parameters. These parameters are called hyperparameters. Better hyperparameters can result in better results(global optima), faster convergence and less over-fitting (by using regularization). Unfortunately, trial and error is the only way to determine the them. Determination of hyper parameter by trial and error is called grid search[4]. To reduce the time taken to determine the best hyperparameters, we increase them in geometric progression.

What are the key components of my approach and results? Also include any specific limitations.

We make four major contributions. First, we use the data set in the running example to train a model using logistic regression using SGD and L-BFGS. We do experiments to compare the running time of Gradient Descent, SGD and L-BFGS. Secondly, we have used regularization and cross-validation to prevent the model over-fitting the data. We report the performance of the model with and without these measures. Thirdly, we also report the accuracy of the model on cross-validation and test data set. Finally, we discuss our results, the strengths and weaknesses of our approach.

\section{Preliminaries}
Let us formalize the arguments presented above. We use a scheme similar to the one discussed in [5]. For the rest of the report we assume that capital letters like $Y, X$ refer to a variable and small letters like $y, x$ refers to their values. We also assume that the outcome of the experiment, $Y$ is dichotomous and can have values ${0,1}$ only, while $X$ can have any real value. Now, let us consider an observation $j$ of the experiment. The outcome of this experiment is denoted by $Y$ and can take any of the two values ${0,1}$. Outcome of an observation $j$ is denoted by $y_{j}$. The features, in the observation $j$, are denoted by a vector $x_{j}=(x_{1}, x_{2},...x_{d})$ where d is the number of dimensions. 
\subsection{Logistic Regression}
We use the principle of conditional likelihood\cite{efron1975efficiency} to calculate $\beta_{1},\beta_{2},...\beta_{m}$ such that the probability 

$P(Y|X) = \prod\limits_{1}^n (p_{i})^{y_{i}}(1-p_{i})^{(1-y_{i})}$ is maximum. In logistical regression, the probability $p_{i}$ takes the following form
\begin{align}
P(y|x)=\dfrac{1}{1+e^{-(\beta_{0}+\sum\limits_{j=1}^n \beta_{j}x_{j})}}
\end{align}
where $\sigma=\dfrac{1}{1+e^{z}}$ is the squashing function\\
and $\beta_i$ is a weight associated with $x_i$

It is hard to do numerical calculations with conditional likelihood because computers run into underflows. Therefore, we maximize the log-likelihood $log(P(y|x)$, instead of the likelihood function. This is called log conditional likelihood(LCL). Therefore, first partial derivative of log-likelihood takes the following form

\begin{align}\label{eq:max}
\dfrac{\partial(log(P(y|x))}{\partial\beta_{j}} = \sum\limits_{i=1}^n(y_i-p_i)x_{ij}
\end{align}

\subsection{Gradient Descent}
We solve equation \ref{eq:max} using gradient descent[6], which takes the following form

\begin{align}\label{eq:betaUpdate}
\beta_j \gets \beta_j + \lambda\sum\limits_{i=1}^n(y_i-p_i)x_{ij}
\end{align}
where $\lambda$ is called learning rate and is independent of the observations

Gradient descent can only be used to maximize a convex function because it converges to the local optima. Since a convex function has only one optima, gradient descent can be used to maximize it. A function is a convex function if its second derivative is always positive. As the equation below shows, log likelihood is a convex function

\begin{align}
\dfrac{\partial^2(log(P(y|x))}{\partial\beta_{j}^2} = -\sum\limits_{i=1}^np_i(y_i-p_i)x_{ij}^2
\end{align}

The equation \ref{eq:betaUpdate} implies that for each $\beta_j$ update, sum over all the observations is required. In practise, this turns out to be slow. Therefore several other methods have been developed to improve the performance of the convergence algorithm. We discuss two such methods used in our experiments,1) SGD 2)L-BFGS.
\\
\subsubsection*{Stochastic Gradient Descent}
SGD proposes that instead of calculating each $\beta_j$ over all the observations, the process can be randomized. Thus, for an observation $i$ the update rule \\ref{eq:betaUpdate} can update all $\beta_j$ once. Due to its random nature, this process is called stochastic gradient descent.
\subsubsection*{Limited Memory Broyden-Fletcher-Goldfarb-Shanno algorithm (L-BFGS}
L-BFGS is a quasi-Newton optimization method which approximates the Hessian matrix to optimize likelihood. We do not go into the details of the algorithm here and readers are encouraged to read \cite{malouf2002comparison}
\subsection{Feature Scaling}
Feature scaling is used standardize the vectors of the observations of an experiment. Feature scaling is generally done to ensure that widely varying vectors can be evaluated at the same scale\cite{bro2003centering}. It also helps in computation as large values of features can cause overflows, without affecting the fit of the data. We scale the input vectors by their Euclidean norm.
\subsection{Over-Fitting}
Overfitting is defined as a characteristic of a regression model which is based on the bias of the sample at hand\cite{babyak2004you}. Overfitting leads to optimistic models which perform well on a training set but fail to produce similar results on a validation set. We discuss two, among many, ways to reduce overfitting, which we have used in our experiments
\subsubsection*{Validation}
The process of validation includes dividing the training set into two parts namely, the training set and the validation set. The training set is used to train the model and the validation set is used to determine the goodness of the fit. A good fit is determined when accuracy of validation set diverges as compared to the training set. The ratio of the number of examples in the validation set is determined by taking into account the number of training set examples which may represent the model adequately and therefore is not standard. Therefore we split the 70\% input data set as training set and the rest as validation set.
\subsubsection*{Regularization}
Regularization is the process of encouraging the parameters to be small especially when the number of observations are smaller than the number of dimensions\cite{ng2004feature}. We use $L_2$ Regularization which subtracts the sum of parameters from the log likelihood. It has a property that it reduces over-fitting as it penalizes the model for large parameter values due to its square term.
\section{Implementation}
Mention that we are not using centering using mean because it is only useful when the a dimension is offset by a constant factor\cite{bro2003centering}. For example, if height is a dimension then, it can be expected that most dimension values will be more than 1m. Since we do not know for sure if that is the case with our data, we do not use centering.
http://www.willamette.edu/~gorr/classes/cs449/overfitting.html
Describe randomization
\subsection{SGD}
\subsection{L-BFGS}
\section{Conclusion}
\section{Results}
1. Comparison of number of iterations, accuracy on training set, accuracy on validation set, accuracy on test set for GD vs SGD vs LGBFS\\
Discussion
2. Comparison of accuracy of training set, validation set, test set for no regularization, regularization, and regularization without intercept\\
Discussion
\section{Content}
Typically, the body of a paper is organized
into a hierarchical structure, with numbered or unnumbered
headings for sections, subsections, sub-subsections, and even
smaller sections.  The command \texttt{{\char'134}section} that
precedes this paragraph is part of such a
hierarchy.\footnote{This is the second footnote.  It
starts a series of three footnotes that add nothing
informational, but just give an idea of how footnotes work
and look. It is a wordy one, just so you see
how a longish one plays out.} \LaTeX\ handles the numbering
and placement of these headings for you, when you use
the appropriate heading commands around the titles
of the headings.  If you want a sub-subsection or
smaller part to be unnumbered in your output, simply append an
asterisk to the command name.  Examples of both
numbered and unnumbered headings will appear throughout the
balance of this sample document.

Because the entire article is contained in
the \textbf{document} environment, you can indicate the
start of a new paragraph with a blank line in your
input file; that is why this sentence forms a separate paragraph.

\subsection{Type Changes and {\subsecit Special} Characters}
We have already seen several typeface changes in this sample.  You
can indicate italicized words or phrases in your text with
the command \texttt{{\char'134}textit}; emboldening with the
command \texttt{{\char'134}textbf}
and typewriter-style (for instance, for computer code) with
\texttt{{\char'134}texttt}.  But remember, you do not
have to indicate typestyle changes when such changes are
part of the \textit{structural} elements of your
article; for instance, the heading of this subsection will
be in a sans serif\footnote{A third footnote, here.
Let's make this a rather short one to
see how it looks.} typeface, but that is handled by the
document class file. Take care with the use
of\footnote{A fourth, and last, footnote.}
the curly braces in typeface changes; they mark
the beginning and end of
the text that is to be in the different typeface.

You can use whatever symbols, accented characters, or
non-English characters you need anywhere in your document;
you can find a complete list of what is
available in the \textit{\LaTeX\
User's Guide}\cite{Lamport:LaTeX}.

\subsection{Math Equations}
You may want to display math equations in three distinct styles:
inline, numbered or non-numbered display.  Each of
the three are discussed in the next sections.

\subsubsection{Inline (In-text) Equations}
A formula that appears in the running text is called an
inline or in-text formula.  It is produced by the
\textbf{math} environment, which can be
invoked with the usual \texttt{{\char'134}begin. . .{\char'134}end}
construction or with the short form \texttt{\$. . .\$}. You
can use any of the symbols and structures,
from $\alpha$ to $\omega$, available in
\LaTeX\cite{Lamport:LaTeX}; this section will simply show a
few examples of in-text equations in context. Notice how
this equation: \begin{math}\lim_{n\rightarrow \infty}x=0\end{math},
set here in in-line math style, looks slightly different when
set in display style.  (See next section).

\subsubsection{Display Equations}
A numbered display equation -- one set off by vertical space
from the text and centered horizontally -- is produced
by the \textbf{equation} environment. An unnumbered display
equation is produced by the \textbf{displaymath} environment.

Again, in either environment, you can use any of the symbols
and structures available in \LaTeX; this section will just
give a couple of examples of display equations in context.
First, consider the equation, shown as an inline equation above:
\begin{equation}\lim_{n\rightarrow \infty}x=0\end{equation}
Notice how it is formatted somewhat differently in
the \textbf{displaymath}
environment.  Now, we'll enter an unnumbered equation:
\begin{displaymath}\sum_{i=0}^{\infty} x + 1\end{displaymath}
and follow it with another numbered equation:
\begin{equation}\sum_{i=0}^{\infty}x_i=\int_{0}^{\pi+2} f\end{equation}
just to demonstrate \LaTeX's able handling of numbering.

\subsection{Citations}
Citations to articles \cite{bowman:reasoning, clark:pct, braams:babel, herlihy:methodology},
conference
proceedings \cite{clark:pct} or books \cite{salas:calculus, Lamport:LaTeX} listed
in the Bibliography section of your
article will occur throughout the text of your article.
You should use BibTeX to automatically produce this bibliography;
you simply need to insert one of several citation commands with
a key of the item cited in the proper location in
the \texttt{.tex} file \cite{Lamport:LaTeX}.
The key is a short reference you invent to uniquely
identify each work; in this sample document, the key is
the first author's surname and a
word from the title.  This identifying key is included
with each item in the \texttt{.bib} file for your article.

The details of the construction of the \texttt{.bib} file
are beyond the scope of this sample document, but more
information can be found in the \textit{Author's Guide},
and exhaustive details in the \textit{\LaTeX\ User's
Guide}\cite{Lamport:LaTeX}.

This article shows only the plainest form
of the citation command, using \texttt{{\char'134}cite}.
This is what is stipulated in the SIGS style specifications.
No other citation format is endorsed.

\subsection{Tables}
Because tables cannot be split across pages, the best
placement for them is typically the top of the page
nearest their initial cite.  To
ensure this proper ``floating'' placement of tables, use the
environment \textbf{table} to enclose the table's contents and
the table caption.  The contents of the table itself must go
in the \textbf{tabular} environment, to
be aligned properly in rows and columns, with the desired
horizontal and vertical rules.  Again, detailed instructions
on \textbf{tabular} material
is found in the \textit{\LaTeX\ User's Guide}.

Immediately following this sentence is the point at which
Table 1 is included in the input file; compare the
placement of the table here with the table in the printed
dvi output of this document.

\begin{table}
\centering
\caption{Frequency of Special Characters}
\begin{tabular}{|c|c|l|} \hline
Non-English or Math&Frequency&Comments\\ \hline
\O & 1 in 1,000& For Swedish names\\ \hline
$\pi$ & 1 in 5& Common in math\\ \hline
\$ & 4 in 5 & Used in business\\ \hline
$\Psi^2_1$ & 1 in 40,000& Unexplained usage\\
\hline\end{tabular}
\end{table}

To set a wider table, which takes up the whole width of
the page's live area, use the environment
\textbf{table*} to enclose the table's contents and
the table caption.  As with a single-column table, this wide
table will ``float" to a location deemed more desirable.
Immediately following this sentence is the point at which
Table 2 is included in the input file; again, it is
instructive to compare the placement of the
table here with the table in the printed dvi
output of this document.


\begin{table*}
\centering
\caption{Some Typical Commands}
\begin{tabular}{|c|c|l|} \hline
Command&A Number&Comments\\ \hline
\texttt{{\char'134}alignauthor} & 100& Author alignment\\ \hline
\texttt{{\char'134}numberofauthors}& 200& Author enumeration\\ \hline
\texttt{{\char'134}table}& 300 & For tables\\ \hline
\texttt{{\char'134}table*}& 400& For wider tables\\ \hline\end{tabular}
\end{table*}
% end the environment with {table*}, NOTE not {table}!

\subsection{Figures}
Like tables, figures cannot be split across pages; the
best placement for them
is typically the top or the bottom of the page nearest
their initial cite.  To ensure this proper ``floating'' placement
of figures, use the environment
\textbf{figure} to enclose the figure and its caption.

This sample document contains examples of \textbf{.eps}
and \textbf{.ps} files to be displayable with \LaTeX.  More
details on each of these is found in the \textit{Author's Guide}.

\begin{figure}
\centering
\epsfig{file=fly.eps, height=1in, width=1in}
\caption{A sample black and white graphic (.eps format)
that has been resized with the \texttt{epsfig} command.}
\end{figure}


As was the case with tables, you may want a figure
that spans two columns.  To do this, and still to
ensure proper ``floating'' placement of tables, use the environment
\textbf{figure*} to enclose the figure and its caption.

Note that either {\textbf{.ps}} or {\textbf{.eps}} formats are
used; use
the \texttt{{\char'134}epsfig} or \texttt{{\char'134}psfig}
commands as appropriate for the different file types.

\subsection{Theorem-like Constructs}
Other common constructs that may occur in your article are
the forms for logical constructs like theorems, axioms,
corollaries and proofs.  There are
two forms, one produced by the
command \texttt{{\char'134}newtheorem} and the
other by the command \texttt{{\char'134}newdef}; perhaps
the clearest and easiest way to distinguish them is
to compare the two in the output of this sample document:

This uses the \textbf{theorem} environment, created by
the\linebreak\texttt{{\char'134}newtheorem} command:
\newtheorem{theorem}{Theorem}
\begin{theorem}
Let $f$ be continuous on $[a,b]$.  If $G$ is
an antiderivative for $f$ on $[a,b]$, then
\begin{displaymath}\int^b_af(t)dt = G(b) - G(a).\end{displaymath}
\end{theorem}

The other uses the \textbf{definition} environment, created
by the \texttt{{\char'134}newdef} command:
\newdef{definition}{Definition}
\begin{definition}
If $z$ is irrational, then by $e^z$ we mean the
unique number which has
logarithm $z$: \begin{displaymath}{\log e^z = z}\end{displaymath}
\end{definition}

Two lists of constructs that use one of these
forms is given in the
\textit{Author's  Guidelines}.

and don't forget to end the environment with
{figure*}, not {figure}!
 
There is one other similar construct environment, which is
already set up
for you; i.e. you must \textit{not} use
a \texttt{{\char'134}newdef} command to
create it: the \textbf{proof} environment.  Here
is a example of its use:
\begin{proof}
Suppose on the contrary there exists a real number $L$ such that
\begin{displaymath}
\lim_{x\rightarrow\infty} \frac{f(x)}{g(x)} = L.
\end{displaymath}
Then
\begin{displaymath}
l=\lim_{x\rightarrow c} f(x)
= \lim_{x\rightarrow c}
\left[ g{x} \cdot \frac{f(x)}{g(x)} \right ]
= \lim_{x\rightarrow c} g(x) \cdot \lim_{x\rightarrow c}
\frac{f(x)}{g(x)} = 0\cdot L = 0,
\end{displaymath}
which contradicts our assumption that $l\neq 0$.
\end{proof}

Complete rules about using these environments and using the
two different creation commands are in the
\textit{Author's Guide}; please consult it for more
detailed instructions.  If you need to use another construct,
not listed therein, which you want to have the same
formatting as the Theorem
or the Definition\cite{salas:calculus} shown above,
use the \texttt{{\char'134}newtheorem} or the
\texttt{{\char'134}newdef} command,
respectively, to create it.

\subsection*{A {\secit Caveat} for the \TeX\ Expert}
Because you have just been given permission to
use the \texttt{{\char'134}newdef} command to create a
new form, you might think you can
use \TeX's \texttt{{\char'134}def} to create a
new command: \textit{Please refrain from doing this!}
Remember that your \LaTeX\ source code is primarily intended
to create camera-ready copy, but may be converted
to other forms -- e.g. HTML. If you inadvertently omit
some or all of the \texttt{{\char'134}def}s recompilation will
be, to say the least, problematic.

\section{Conclusions}
This paragraph will end the body of this sample document.
Remember that you might still have Acknowledgments or
Appendices; brief samples of these
follow.  There is still the Bibliography to deal with; and
we will make a disclaimer about that here: with the exception
of the reference to the \LaTeX\ book, the citations in
this paper are to articles which have nothing to
do with the present subject and are used as
examples only.
%\end{document}  % This is where a 'short' article might terminate

%ACKNOWLEDGMENTS are optional
\section{Acknowledgments}
This section is optional; it is a location for you
to acknowledge grants, funding, editing assistance and
what have you.  In the present case, for example, the
authors would like to thank Gerald Murray of ACM for
his help in codifying this \textit{Author's Guide}
and the \textbf{.cls} and \textbf{.tex} files that it describes.

%
% The following two commands are all you need in the
% initial runs of your .tex file to
% produce the bibliography for the citations in your paper.
\bibliographystyle{abbrv}
\bibliography{Report}  % sigproc.bib is the name of the Bibliography in this case
% You must have a proper ".bib" file
%  and remember to run:
% latex bibtex latex latex
% to resolve all references
%
% ACM needs 'a single self-contained file'!
%
%APPENDICES are optional
%\balancecolumns
\appendix
%Appendix A
\section{Headings in Appendices}
The rules about hierarchical headings discussed above for
the body of the article are different in the appendices.
In the \textbf{appendix} environment, the command
\textbf{section} is used to
indicate the start of each Appendix, with alphabetic order
designation (i.e. the first is A, the second B, etc.) and
a title (if you include one).  So, if you need
hierarchical structure
\textit{within} an Appendix, start with \textbf{subsection} as the
highest level. Here is an outline of the body of this
document in Appendix-appropriate form:
\subsection{Introduction}
\subsection{The Body of the Paper}
\subsubsection{Type Changes and  Special Characters}
\subsubsection{Math Equations}
\paragraph{Inline (In-text) Equations}
\paragraph{Display Equations}
\subsubsection{Citations}
\subsubsection{Tables}
\subsubsection{Figures}
\subsubsection{Theorem-like Constructs}
\subsubsection*{A Caveat for the \TeX\ Expert}
\subsection{Conclusions}
\subsection{Acknowledgments}
\subsection{Additional Authors}
This section is inserted by \LaTeX; you do not insert it.
You just add the names and information in the
\texttt{{\char'134}additionalauthors} command at the start
of the document.
\subsection{References}
Generated by bibtex from your ~.bib file.  Run latex,
then bibtex, then latex twice (to resolve references)
to create the ~.bbl file.  Insert that ~.bbl file into
the .tex source file and comment out
the command \texttt{{\char'134}thebibliography}.
% This next section command marks the start of
% Appendix B, and does not continue the present hierarchy
\section{More Help for the Hardy}
The acm\_proc\_article-sp document class file itself is chock-full of succinct
and helpful comments.  If you consider yourself a moderately
experienced to expert user of \LaTeX, you may find reading
it useful but please remember not to change it.
\balancecolumns
% That's all folks!
\end{document}
